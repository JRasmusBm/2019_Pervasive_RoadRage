\section{Research}\label{sec:research}

The beginning of the project featured an extended research
period, where the original research question was studied
in depth, as well as ideas for the project at hand. In
Sections~\ref{sec:research_question}~to~\ref{sec:choice_of_technologies}
the results of that research are documented. Then in
Sections~\ref{sec:plan_of_approach}~to~\ref{sec:amended_research_question}
the plan of approach for the project is laid out and the research question
is amended.

\subsection{Original Research Question}\label{sec:research_question}

Can collisions be avoided by monitoring the driver's stress levels?

\subsection{Measuring Stress Response of
Drivers}\label{sec:research_stress}

In \textit{Road traffic offending, stress, age, and accident history among male
and female drivers}~\cite{road_traffic} it is stated that stress is detremental
to driver performance. In \textit{A Critical Review of Proactive Detection of
Driver Stress Levels Based on Multimodal Measurements}~\cite{critical_review}
it is
also stated that unsuccessful driver performance causes over 80\% of vehicle
crashes that result in fatalities and injuries. The latter further confirms
that driver performance can be negatively affected by stress which leads to
both high mental workload and negative moods in drivers and may result in
inadequate information processing and imperfect perception.

It also mentions that, among many others ways to physiologically monitor
stress, HR (Heart Rate) and GSR (also known as EDA or SCL, respectively
Galvanic Skin Response, Electro-Dermal Activity and Skin Conductance Level)
are among the most reliable indicators of driver stress. However, they
insist on the fact that inter-subject variability between drivers can lead
to fluctuations in the values of features, such that a machine learning
approach of the subject is highly recommended.

Lessons were also learned from \textit{Detecting Stress During Real-World
Driving Tasks Using Physiological Sensors}~\cite{detecting_stress}, in which
stress was monitored throughout a real world-driving task. What was inspiring
about the stress paper is they had 2 different approaches: the first idea was
to observe the sensors results for 5-minute intervals and then determining
a general level over those 5 minutes and the other was measuring stress
level based on a real-time observation of the sensors' outputs. It, again,
agrees that HR and GSR are likely to be the best overall physiological ways
to detect stress.

The findings in \textit{EEG alpha asymmetry, heart rate variability and
cortisol in response to virtual reality induced stress}~\cite{eeg_alpha}
were very useful, because it discussed about the best real-time indicator
of stress in the case of a virtual reality induced one. In particular,
it showed that HR (and especially HR variability) might be the best bet
even if it would require quite large windows of analysis. Unfortunately,
they didn't take GSR as an eventual stress correlated feature.

\subsection{Creating a Virtual Environment}\label{sec:research_vr}

In \textit{Immersion factors affecting perception and behavior in a virtual
reality power wheelchair simulator}~\cite{immersion_factors} it was found that
some kind of VR headset improved the sense of immersion. 

It was also noted that in their attempts, they used a very bland environment
to focus specifically on the measurements they were trying to gather. In the
case of this project, the opposite was the case. The goal was to induce
feelings akin to those of driving a real car, so as much reality as possible
was desired.

\subsection{Choice of Technologies}\label{sec:choice_of_technologies}

As explained in Section~\ref{sec:research_vr}, VR was strongly preferred
for immersion. For specific VR equipment, it was decided use Oculus Go
in combination with Unity for the virtual environment, as those seemed to
have the most streamlined interface and the scene was not expected to be
very complicated.

It was decided to use the Shimmer3 GSR+ Unit along with two GSR electrodes
(fingers) and an optical pulse ear-clip all from Shimmer. During the
research it was found, with regards to the stress detection system, that
GSR was preferred over HR (See Section~\ref{sec:research_stress}). It was
thus decided to let the optical pulse go. A C\# API from Shimmer's website
was provided communicate with the GSR+ Unit.

A miniature remote controlled car was chosen (as it was not within the budget
of this project to purchase a real car) with an Arduino attached to control
the motors. A Raspberry Pi 3 B+ was attached to the Arduino as it alone does
not support connections over WiFi.

The Raspberry Pi was programmed using Python for socket connections to the
client PC used for stress measurements and steering. This communicates with
both C\# and Python clients on the PC for stress measurements and driving
control respectively.

\subsection{Plan of Approach}\label{sec:plan_of_approach}

At the end of the research phase it was determined to revisit the original
research question and set concrete goals for the project. 

It was decided that, though in reality perhaps infeasible, it would still
be attempted to let the stress level directly affect car speed. For this system
an autonomous model car would be co-opted for remote control. Stress levels
would be measured using a Shimmer3 device.

The project would be considered a success if the end system included a remote
controlled car with some interaction with the Skin Conductance sensor, as at
this point any level of interaction between driving controls and user stress
was considered quite a feat.  The design plan will be further described in
Section~\ref{sec:design}.

\subsection{Research Question}\label{sec:amended_research_question}

Can remote controlled car performance be improved by monitoring the user's
stress levels?
