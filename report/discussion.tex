\section{Discussion}\label{sec:discussion}

Road security is an important way of saving lives and property. The smart
cars industry is always trying to make safer cars. New generations of smart
cars can focus on the driver performance, and make the motor world safer. A
monitor for the driver's emotional state can warn the driver of driving in
a bad condition, or even assist in controlling the car. With this project
it was attempted to provide a baseline to adjust to the stress, and make
the driving easier when the situation requires it.

In the project, it was possible to detect the driver stress of a
radio controlled car. The car top speed slows down when the stress is
elevated. It makes easier to control the remote controlled car. The system
works personalized for each driver, applying a k\-means algorithm with the
skin conductance sensor. So it is more reliable detecting the stress of
each driver, with different skin characteristics. With the VR environment,
the sensation of driving a real car is induced. With all these factors,
the project is considered to be a success. 

In terms of the work done, getting and connecting all the devices proved more
difficult than expected. This kind of the project requires the inter-operation
of many devices. All the connections and work between them had to be efficient
enough for the whole system to transparently work as one. It represented
a challenge in the very beginning of the design and implementation work.

The skin conductance technology has been proven to be reliable enough to
measure stress response. Improving the k\-means algorithm of the sensor,
making it more efficient is an important part of any future work. Using heart
rate variability could also present an improvement to the current performance.

The system could be improved by outsourcing rendering for the headset
to another device, such that the hardware limitations become less of a
constraint. This would yield a smoother video experience while wearing the
Oculus GO\@. Another possible improvement relates to the fact that the driver
is not informed of their state of stress, but the speed of the car. Stress
indicators in the VR environment can be used in future projects to amend
this discrepancy.

The main challenge with further implementation and future work is that the
stress level usually peaks after the driver has already had a stressful
encounter in traffic. This means that their stress level can only be used as
guiding data to the information system to the car, as it cannot be relied
upon to safely influence the actual driving system. Thus future work would
revolve around finding unobtrusive ways to help the driver focus, make smart
decisions and ultimately calm down when stressed.
