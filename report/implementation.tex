\section{Resulting System}\label{sec:implementation}

The entire system was able to be implemented as described in
Section~\ref{sec:design}.  In this section challenges will be documented that
were encountered and overcome in the implementation stage of the project. These
included hardware limitations (See Section~\ref{sec:hardware_limitations}),
power supply (See Section~\ref{sec:power_supply}) and the non-standard
Do-It-Yourself (DIY) work needed to mount the technology onto the car.

\subsection{Hardware limitations}\label{sec:hardware_limitations}

As the project progressed multiple hardware limitations were found, some
impacted the feasibility of the project or could otherwise relatively easily
be circumvented and others were determined to be minor enough to ignore.
The Raspberry Pi camera has a relatively low Field of View (FOV), resulting
in a difficult to approximate depth for the user.


The initial Raspberry Pi, a model Zero-W, which has a single core that was
continuously under 100\% load during use, resulting in stuttering of the video
stream. The Raspberry Pi Zero-W also has inferior networking capabilities
in comparison to the bigger 3B+ which was also available.  For these reasons
it was decided to replace the Zero-W with a 3B+.

The initial router used was a ZyXeL P-335U which in use was found to have
insufficient range and speed. After asking the project supervisor, a slightly
newer router, a D-Link DIR-635, was borrowed which significantly improved
the user experience in subsequent tests.

While developing it was found that remote control car, controlled by an
Arduino, was not able to reverse without first coming to a complete stop
by writing the corresponding stop value to the throttle pin. This made
for a non-intuitive driving experience as one would have to let go of the
controls for a significant amount of time before being able to reverse.
This was alleviated by continuously writing the corresponding stop and
reverse values to the pin, causing a stuttering effect while reversing but
solving the primary problem.

The Oculus Go renders everything on it's own hardware, limiting the performance
of the virtual environment created to its capabilities.

\subsection{Power Supply}\label{sec:power_supply}

Initially powering the Raspberry Pi was intended to be done through the Arduino
or directly from the battery pack that powers it and the remote control car.

After some consideration it was decided that to power the Raspberry Pi through
a power-bank,  as the battery pack plugs directly into the car using a 2-pin
Molex connector and any attempt to power the Raspberry Pi from there would
require additional connectors and cables. It would also risk overloading
the Arduino power regulator, as it is not intended to simultaneously power
multiple devices on that scale.

\subsection{DIY Mounting Solutions}\label{sec:diy}

The system was assembled in steps as requirements arose. As time was limited
and the requirements, and subsequently the equipment, could change in a day,
permanent or otherwise more time consuming mounting solutions were discarded. 

Whether the time required would've been waiting for 3d-printing to finish or
delivery of ordered products.  

The solutions used primarily used electrical tape and stray pieces of
cardboard. The specific implementations were:
\begin{itemize}
\item A powerbank for powering the Raspberry Pi 3 taped to the fender. 
\item A Raspberry Pi 3 for primary user communication taped onto the back of
the car. 
\item A Raspberry Pi camera taped to a piece of cardboard which itself is
fastened with tape to be just above the Raspberry Pi 3.
\end{itemize}
%
% Evaluate the result based on the requirements of success outlined in
% ~\ref{sec:plan_of_approach} 
